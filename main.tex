\documentclass{article}
\usepackage[utf8]{inputenc}

\title{Requirements Document for Amac Repair Tool\\ First Draft}
\author{Ruben Langius - S2583992}
\date{September 2017}

\begin{document}

\maketitle

\section{Introduction}
Amac is a Apple Premium Reseller in the netherland. They aim to give the best possible Apple experience. One of their biggest problems is proper management of their repair processes in which they repair the devices of customers. There are several tools in use but they would like to have a clear overview of the requirements of one system that cover all the aspects.

\subsection{Company Information}
Amac is the biggest Apple Premium Reseller in the Netherlands. With 46 shops and a web-shop they provide their customers with a great shopping experience. Amac is also an authorized repair partner from Apple. Using one central repair centre in Utrecht and several smaller in-shop repair departments they are able to give their customers the hardware support they need.\\
\\
My main contact person was the store manager from Amac Groningen, Boaz Molenaar. But I've interviewed multiple people which I all list below. 

\begin{enumerate}
\item Boaz Molenaar, Store manager \\
email: b.molenaar@amac.nl\\
tel : +31 6 52 37 17 03
\item John Buijs, Repair technician 
\item Bjorn Appeldoorn, Salesman
\end{enumerate}


\subsection{Purpose}
The purpose of this document is to build a repair tool to manage the whole repair process. This includes the customer communication, ordering parts, legal and payment processes. 

\subsection{Scope}
The repair tool will be used to manage all the repair processes within Amac. The system will be designed to maximize the productivity for all the users by automating as much as possible. The system will facilitate communication between customer, salesman, repair technicians and managers. All the functionality will be implemented in a format which is easy to understand and use.


\section{Product Overview}

A repair management system stores the following information.\\
\\
Customer details:\\
It includes customer code, name, address, email and phone number. This information is used to communicate with the customer. It may also be used for future marketing purposes.\\
\\
Product details:\\
It includes serial number, type of product, current state and personal information like passwords if applicable.\\
\\
Repair details:\\
It includes customer problem, internal notes from employees, ordered parts and costs / payment information.\\
\\


\subsection {Users and Stakeholders}
Users:
\begin{enumerate}
    \item Salesman
    \item Repair technician
    \item Management
    \item Customers
\end{enumerate}
\\
Customers:
\begin{enumerate}
    \item Amac B.V.
    \item An Amac Customer with a repair.
\end{enumerate}


\subsection{User Characteristics}
The system will be used in a wide variety of enviroment but most of all within Amac stores / headquarter.  The salesman, technicians, management and customer will be the main users. Given the condition that not all the users are computer-literate. Some users may have to be trained on using the system. The system is also designed to be user-friendly. It uses a Graphical User Interface (GUI). \\
\\
Salesman:\\ 
They all have general sales duties. Every staff is experienced with computers, they are already familiar with the systems currently used. The sales employees are responsible for the check-in of the repair, in which all the information about the customer and the product is entered and validated. \\
\\
Repair technician:\\
The repair technician are responsible for fixing any software or hardware related issues. Due to their techincal skill, they are already familiar with complicated systems. In the repair tool these employees are responsible for keeping track of the current process of all the repairs, including proper communication to the customer.\\
\\
Management:\\
Management is responsible for keeping track of all the activities and managing the shops in such a way that every shop is most efficient and returns the best possible customer experience possible. For these tasks they will use the repair tool to gather statistics. \\
\\
Customers:\\
The customer will be able to track their repair process and accept any offers send by the repair technician. The experience of the customers with these system is higly variable. \\


\subsection{Use cases}
These are the use cases for the users of my App
\begin{enumerate}
    \item Creating a repair with the customer
    \item Updating state of repairs
    \item Ordering parts needed for the repairs
    \item Automatic Customer Satisfaction reports.
    \item Generating overall Statistics.
\end{enumerate}

\section{Functional Requirements}
\subsection{Register Customer}
Salesman should be able to register the customer.
\subsection{Gathering Product Info}
While creating a repair, the salesman must be able to gather product information using the serial number from the broken device.
\subsection{Communication}
Repair staff should be able to communicate with the customer from inside the tool during the repair

\section{Non Functional Requirements} 
\subsection{Privacy}
Customer information should be handled with great care. Especially information regarding Apple ID's Best is to implement encryption for all data. 
\subsection{Performance}
To create the best customer experience, speed is of essence. Especially the salesman in the shops are busy with multiple customers at once. Therefore any delay in the tool is not acceptable. 

\end{document}
